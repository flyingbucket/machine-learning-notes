\documentclass[a4paper,12pt]{report}
\usepackage[margin=1in]{geometry} % to change the page dimensions
\usepackage{ctex}
\usepackage{xeCJK}
\usepackage{comment}
\usepackage{amsmath}
\usepackage{amssymb}
\usepackage{amsthm}
%\usepackage{times}
\usepackage{setspace}
% \usepackage{lastpage}
\usepackage{fancyhdr}
\usepackage{graphicx}
%\graphicspath{{fig/}}
\usepackage{wrapfig}
\usepackage{subfigure}
\usepackage{array}  
% \usepackage{fontspec,xunicode,xltxtra}
% \renewcommand{\sfdefault}{cmr}
\usepackage{titlesec}
\usepackage{titletoc}
\usepackage[titletoc]{appendix}
%\usepackage[top=30mm,bottom=30mm,left=20mm,right=20mm]{geometry}
%\usepackage{cite}
\usepackage[backend = biber, style = gb7714-2015, defernumbers=true]{biblatex}
\renewcommand*{\bibfont}{\small}
\addbibresource{reference.bib}
%\usepackage{courier}
\setmonofont{Courier New}
\usepackage{listings}
\lstset{tabsize=4, keepspaces=true,
    xleftmargin=2em,xrightmargin=0em, aboveskip=1em,
    %backgroundcolor=\color{gray!20},  % 定义背景颜色
    frame=none,                       % 表示不要边框
    extendedchars=false,              % 解决代码跨页时,章节标题,页眉等汉字不显示的问题
    numberstyle=\ttfamily,
    basicstyle=\ttfamily,
    keywordstyle=\color{blue}\bfseries,
    breakindent=10pt,
    identifierstyle=,                 % nothing happens
    commentstyle=\color{green}\small,  % 注释的设置
    morecomment=[l][\color{green}]{\#},
    numbers=left,stepnumber=1,numberstyle=\scriptsize,
    showstringspaces=false,
    showspaces=false,
    flexiblecolumns=true,
    breaklines=true, breakautoindent=true,breakindent=4em,
    escapeinside={/*@}{@*/},
}
\usepackage{amsmath}
\usepackage{amsthm}
\newtheorem{theorem}{定理}
\newtheorem{definition}{定义}
\newtheorem{corollary}{推论}
\newtheorem{example}{例}
\usepackage{amsfonts}
%\usepackage{bm}
\usepackage{booktabs} % for much better looking tables
\usepackage{paralist} % very flexible & customisable lists (eg. enumerate/itemize, etc.)
\usepackage{verbatim} % adds environment for commenting out blocks of text & for better verbatim
\usepackage{subfigure} % make it possible to include more than one captioned figure/table in a single float
% These packages are all incorporated in the memoir class to one degree or another...
\usepackage{cases} %equation set
\usepackage{multirow} %use table
\usepackage{algorithm}
\usepackage{algorithmic}
\usepackage{hyperref}
\hypersetup{colorlinks,linkcolor=black,anchorcolor=black,citecolor=black, pdfstartview=FitH,bookmarksnumbered=true,bookmarksopen=true,} % set href in tex & pdf
%\usepackage[framed,numbered,autolinebreaks,useliterate]{mcode} % 插入matlab代码
\XeTeXlinebreaklocale "zh"
\XeTeXlinebreakskip = 0pt plus 1pt minus 0.1pt

\makeatletter
\newenvironment{breakablealgorithm}
  {% \begin{breakablealgorithm}
   \begin{center}
     \refstepcounter{algorithm}% New algorithm
     \hrule height.8pt depth0pt \kern2pt% \@fs@pre for \@fs@ruled
     \renewcommand{\caption}[2][\relax]{% Make a new \caption
       {\raggedright\textbf{\ALG@name~\thealgorithm} ##2\par}%
       \ifx\relax##1\relax % #1 is \relax
         \addcontentsline{loa}{algorithm}{\protect\numberline{\thealgorithm}##2}%
       \else % #1 is not \relax
         \addcontentsline{loa}{algorithm}{\protect\numberline{\thealgorithm}##1}%
       \fi
       \kern2pt\hrule\kern2pt
     }
  }{% \end{breakablealgorithm}
     \kern2pt\hrule\relax% \@fs@post for \@fs@ruled
   \end{center}
  }
\makeatother
%%%%此处break 算法
%---------------------------------------------------------------------
%	页眉页脚设置
%---------------------------------------------------------------------
\fancypagestyle{plain}{
    \pagestyle{fancy}      %改变章节首页页眉
}

\pagestyle{fancy}
\lhead{\kaishu~数据挖掘实验报告~}
%\rhead{\kaishu~xxx}
\cfoot{\thepage}
\titleformat{\chapter}{\centering\zihao{2}\heiti}{第\chinese{chapter}章}{1em}{}
% \titleformat{\chapter*}{\centering\zihao{-1}\heiti}
\begin{comment}
%---------------------------------------------------------------------
%	章节标题设置
%---------------------------------------------------------------------
\titleformat{\chapter}{\centering\zihao{-1}\heiti}{实验\chinese{chapter}}{1em}{}
\titlespacing{\chapter}{0pt}{*0}{*6}
\end{comment}
%---------------------------------------------------------------------
%	摘要标题设置
%---------------------------------------------------------------------
%\renewcommand{\abstractname}{摘要}
\renewcommand{\figurename}{图}
\renewcommand{\tablename}{表}

%---------------------------------------------------------------------
%	参考文献设置
%---------------------------------------------------------------------
%\renewcommand{\bibname}{\zihao{2}{\hspace{\fill}参\hspace{0.5em}考\hspace{0.5em}文\hspace{0.5em}献\hspace{\fill}}}
\renewcommand{\bibname}{参考文献}
\begin{comment}
%---------------------------------------------------------------------
%	引用文献设置为上标
%---------------------------------------------------------------------
\makeatletter
\def\@cite#1#2{\textsuperscript{[{#1\if@tempswa , #2\fi}]}}
\makeatother
\end{comment}
%---------------------------------------------------------------------
%	目录页设置
%---------------------------------------------------------------------
%\renewcommand{\contentsname}{\zihao{-3} 目\quad 录}
\renewcommand{\contentsname}{目录}
\titlecontents{chapter}[0em]{\songti\zihao{-4}}{\thecontentslabel\ }{}
{\hspace{.5em}\titlerule*[4pt]{$\cdot$}\contentspage}
\titlecontents{section}[2em]{\vspace{0.1\baselineskip}\songti\zihao{-4}}{\thecontentslabel\ }{}
{\hspace{.5em}\titlerule*[4pt]{$\cdot$}\contentspage}
\titlecontents{subsection}[4em]{\vspace{0.1\baselineskip}\songti\zihao{-4}}{\thecontentslabel\ }{}
{\hspace{.5em}\titlerule*[4pt]{$\cdot$}\contentspage}

\begin{document}
%---------------------------------------------------------------------
%	封面设置
%---------------------------------------------------------------------
\begin{titlepage}
    \begin{center}
        
    \includegraphics[width=0.60\textwidth]{nk_logo.pdf}\\
    \vspace{10mm}
    \hspace*{\fill} \\
    \textbf{\zihao{1}{数据挖掘实验报告}}\\
    \vspace{\fill}
    
\setlength{\extrarowheight}{3mm}
{\songti\zihao{3}	
\begin{tabular}{rl}
    
    {\makebox[4\ccwd][s]{学\qquad 号:}} & ~\kaishu   \\
    {\makebox[4\ccwd][s]{姓\qquad 名:}} & ~\kaishu   \\
    {\makebox[4\ccwd][s]{年\qquad 级:}} & ~\kaishu   \\
    {\makebox[4\ccwd][s]{学\qquad 院:}} & ~\kaishu   \\
    {\makebox[4\ccwd][s]{专\qquad 业:}} & ~\kaishu   \\
    %{\makebox[4\ccwd][s]{授课教师:}}  & ~\kaishu xxx~教授\\ 
    %{\makebox[4\ccwd][s]{课程助教:}} & ~\kaishu xxx~xxx \\
    {\makebox[4\ccwd][s]{完成日期:}}  & ~\kaishu 2021年3月27日\\ 

\end{tabular}
 }\\[2cm]
%\vspace{\fill}
%\zihao{4}
%使用\LaTeX 撰写于\today
    \end{center}	
\end{titlepage}

%---------------------------------------------------------------------
%  摘要页
%---------------------------------------------------------------------


%---------------------------------------------------------------------
%  目录页
%---------------------------------------------------------------------
\tableofcontents % 生成目录

%---------------------------------------------------------------------
%  绪论
%---------------------------------------------------------------------
\chapter{第一次上机实验(LBP提取图像的纹理特征)}
\section{实验要求}
\begin{itemize}
\item{1. 给定若干张图像,利用局部二值模式特征(LBP)对这些图像进行特征提取}
\item{2. 图象是W * H * 3的矩阵}
\item{3. 将最终提取到的特征通过plot的形式展示,绘制特征曲线图直观对比不同类图片纹理提取到的特征的不同}
\item{4. 使用Python编程实现}
\end{itemize}
% \section{数据分析与处理}
% \par 没有可以不写,比如垂直平分分类器中
\section{实验步骤与原理}
\subsection{LBP 特征的基本定义}
局部二值模式(Local Binary Pattern, LBP)通过比较像素与其邻域像素的灰度关系,
编码局部纹理的微结构。给定中心像素 $g_c$ 及以其为中心、
半径为 $R$ 的圆形邻域上 $P$ 个等角度采样点的灰度 
$\{g_p\}_{p=0}^{P-1}$,标准 LBP 的定义为
\[
\mathrm{LBP}_{P,R}(x_c,y_c)
=\sum_{p=0}^{P-1} s(g_p-g_c)\,2^{p},\qquad
s(t)=\begin{cases}
1,& t\ge 0,\\
0,& t<0,
\end{cases}
\]
其中邻域采样点坐标为
\[
(x_p, y_p) = \bigl(x_c + R\cos(2\pi p/P),\; y_c - R\sin(2\pi p/P)\bigr),
\]
本次实验只考虑以$g_c$为中心的九宫格的局部的LBP特征。

\subsection{直方图特征}
将整幅图像(或图像块)内的 LBP 代码统计为直方图作为纹理特征:
\[
H[k]=\sum_{(x,y)} \mathbf{1}\{\mathrm{LBP}_{P,R}(x,y)=k\},\qquad k\in\{0,\dots,2^P-1\}.
\]
常见做法是对直方图进行 $\ell_1$ 归一化以消除尺寸影响:
\[
\hat H[k]=\frac{H[k]}{\sum_{j} H[j]}.
\]
为表征空间布局,可将图像划分为 $M\times N$ 个网格单元,分别计算直方图并按行优先串接,得到最终特征向量。


\subsection{实现细节(本实验手写 Python 要点)}
\begin{enumerate}
  \item \textbf{预处理:} 彩色图像先转灰度;可选高斯平滑抑制噪声。
  \item{按上述规则计算出图片的LBP特征直方图}
  \item \textbf{可视化:} 使用 Matplotlib 绘制折线;多类对比时可叠加均值曲线与标准差带。
\end{enumerate}

\subsection{复杂度与并行优化}
\begin{itemize}
  \item{时间复杂度约为 $O(PWH)$,$W,H$ 为图像宽高;$P$ 通常较小,易于并行/向量化。}
  \item{下面所呈现的代码采用串行方式计算LBP特征,但本人也给出了基于cython的并行加速版本。}
\end{itemize}

\noindent\textbf{加速计算技巧:}
\begin{itemize}
  \item{使用cython 的memoryview接口直接操作numpy.ndarray}
  \item{在cython层开启python的nogil模式,绕开全局解释器锁,使用OpenMP实现并行计算}
\end{itemize}

并行计算代码及各类计算方法的benchmark详见

\url{https://github.com/flyingbucket/machinelearning/tree/main/LBP}。
\section{实验结果与分析}
\begin{figure}[H]
    \centering
    \includegraphics[width=0.7\textwidth]{../images/cloud_lbp_curves.png}
    \caption{cloud LBP 特征曲线对比图}
    \label{fig:cloud_lbp_curve}
\end{figure}

\begin{figure}[H]
    \centering
    \includegraphics[width=0.7\textwidth]{../images/forest_lbp_curves.png}
    \caption{forestLBP 特征曲线对比图}
    \label{forest_lbp_curve}
\end{figure}

\section{实验代码}
\begin{lstlisting}[language=Python]
import numpy as np
from matplotlib import pyplot as plt
from collections import Counter
from PIL import Image

class LBP:
    @staticmethod
    def _read_img(imPath: str, pad: int = 1, mode: str = "reflect") -> np.ndarray:
        im = Image.open(imPath).convert("L")
        arr = np.array(im)
        padded = np.pad(arr, pad_width=((pad, pad), (pad, pad)), mode=mode)
        return padded

    @staticmethod
    def LBPkernel(im: np.ndarray, x, y) -> int:
        h, w = im.shape
        assert x + 2 < h and y + 2 < w, (
            f"Index out of bound,please check padding. x:{x},y:{y},h:{h},w:{w}"
        )
        patch = im[x : x + 3, y : y + 3].copy()
        patch = (patch >= patch[1, 1]).astype(np.uint8)
        idxs = [0, 1, 2, 5, 8, 7, 6, 3]
        bits = patch.reshape(-1)[idxs]
        val = int("".join(map(str, bits)), 2)
        return val

def walk_dir(root_dir: str, out_dir: str = "EX1/outputs"):
    root = Path(root_dir)
    out = Path(out_dir)
    out.mkdir(parents=True, exist_ok=True)
    LBPcyExecutor = LBP()

    for class_dir in sorted([p for p in root.iterdir() if p.is_dir()]):
        hist_list = []
        img_names = []
        all_codes = set()
        for img_path in sorted(class_dir.iterdir()):
            try:
                res_dict = LBPcyExecutor(str(img_path))  # {code: count}
                if not isinstance(res_dict, dict) or len(res_dict) == 0:
                    print(f"[WARN] 空直方图:{img_path}")
                    continue
                hist_list.append(res_dict)
                img_names.append(img_path.stem)
                all_codes.update(res_dict.keys())
            except Exception as e:
                print(f"[WARN] 处理失败: {img_path} -> {e}")

        codes = sorted(all_codes)  # 所有出现过的 LBP code
        X = []  # 每张图对齐后的频率向量

        for h in hist_list:
            vec = np.array([h.get(c, 0) for c in codes], dtype=np.float64)
            X.append(vec)

        plt.figure(figsize=(10, 6))
        for vec, name in zip(X, img_names):
            plt.plot(codes, vec, linewidth=1.2, alpha=0.85, label=name)
        plt.xlabel("LBP code")
        plt.ylabel("count")
        plt.title(f"LBP feature curves - {class_dir.name}")
        plt.legend(ncol=2, fontsize=9, loc="best")
        plt.tight_layout()

        save_path = out / f"{class_dir.name}_lbp_curves.png"
        plt.savefig(save_path, dpi=160)
        plt.close()
        print(f"[OK] Saved: {save_path}")

if __name__ == "__main__":
    dir = "./EX1/data"
    walk_dir(dir)

\end{lstlisting}
\clearpage
\chapter{第二次上机实验}
\section{实验要求}
\section{数据分析与处理}
\section{实验步骤与原理}
\section{实验结论与分析}
\section{实验代码}
\clearpage
\chapter{第三次上机实验}
\section{实验要求}
\section{数据分析与处理}
\section{实验步骤与原理}
\section{实验结论与分析}
\section{实验代码}
\clearpage
\chapter{第四次上机实验}
\section{实验要求}
\section{数据分析与处理}
\section{实验步骤与原理}
\section{实验结论与分析}
\section{实验代码}
\clearpage
\chapter{第五次上机实验}
\section{实验要求}
\section{数据分析与处理}
\section{实验步骤与原理}
\section{实验结论与分析}
\section{实验代码}
\printbibliography

\end{document}
